\documentclass{article}

\usepackage[top=0.5in, bottom=0.5in, left=0.5in, right=0.5in]{geometry}
\usepackage{enumitem}
\usepackage{hyperref}

\begin{document}
	\begin{center}
		\thispagestyle{empty}
		\large \textbf{Jesús Molina Roldan\\}
		\normalsize jmr\textunderscore sultan@hotmail.com $\mid$ 697656469 $\mid$ \href{https://jesu20950.github.io/}{jesu20950.github.io} \\ 
		\hrulefill
	\end{center}
	
	\noindent \textbf{\underline{EXPERIENCIA LABORAL}} \\
	\noindent \textbf{Everis} \hfill Barcelona, España \\
	\textit{Desarrollador Full Stack, Departamento de seguros} \hfill 2019-2020 \\
	Estuve como becario trabajando con la compañía Mutua de Propietarios. Mis tareas consistieron en la programación y testeo de servicios en una página web.
	\begin{itemize}[noitemsep,nolistsep,leftmargin=*]
		\item {\textbf{Lenguajes de programación:} Java, JavaScript, SQL, HTML.}\\
	\end{itemize}
	

	\noindent \textbf{\underline{EDUCACIÓN}} \\
	\textbf{Maristes Champagnat de Badalona} \\
	\textit{Bachillerato + Tecnológico}  \hfill 2014-2016  \\ \\
	\textbf{Universidad Politécnica de Cataluña} \\
	\textit{Grado en Ingeniería Informática + Computación}  \hfill 2016-Presente \\
	

	\noindent \textbf{\underline{HABILIDADES TÉCNICAS}} \\
	\begin{itemize}[noitemsep,nolistsep,leftmargin=*]
		\item {\textbf{Sistema operativos:} Windows y SOs con Linux.}
		\item {\textbf{Lenguajes de programación:} C, C++, Matlab, Java, JavaScript, SQL, HTML, CSS}
		\item {\textbf{Data Science:} R, Python con NumPy y Pandas}
		\item {\textbf{Control de versiones:} Git, Github}\\
	\end{itemize}
	
		\noindent \textbf{\underline{PROYECTOS}} \\
	\noindent \textbf{qtCrypto}  \hfill  Present \\
	Aplicación para cifrar/descifrar archivos creada con Qt y C++. \\
	\noindent \textbf{Personal website}  \\
	Sitio web personal creado con HTML, JavaScript y CSS. \\
	\noindent \textbf{Chess puzzles}  \\
	Software para manejar problemas de ajedrez creado con Java. \\
	\noindent \textbf{Digit recognition}  \\
	Programa de reconocimiento de dígitos creado con Matlab. \\
	\noindent \textbf{Data compressor}  \\
	Programas para comprimir/descomprimir archivos creados con Python. \\
	
	
	\noindent \textbf{\underline{IDIOMAS}} \\
	\begin{itemize}[noitemsep,nolistsep,leftmargin=*]
		\item {Castellano.}
		\item {Catalán.}
		\item {Inglés. \\}
	\end{itemize}
	

	\noindent \textbf{\underline{SECCIÓN EXTRA}} \\
	\noindent \textbf{Música:} Guitarrista. \\
	\noindent Disponibilidad inmediata. 
	
	
\end{document}